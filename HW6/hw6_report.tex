\documentclass{article}
\usepackage[letterpaper]{geometry}
\geometry{verbose,tmargin=1in,bmargin=1in,lmargin=1in,rmargin=1in}

\usepackage[utf8]{inputenc}
\usepackage{amsmath}
\usepackage{float}
\usepackage{listings}
\usepackage{graphicx}
\usepackage{enumitem}
\usepackage[bbgreekl]{mathbbol}

\renewcommand{\baselinestretch}{1.5}

\title{CIS 419/519: Homework 6}
\author{\{Yupeng Li\}}
\date{04.15.2020}

\begin{document}
    \maketitle
    Although the solutions are entirely my own, I consulted with the following people and sources while working on this homework: \textbf{https://www.youtube.com/watch?v=kNPGXgzxoHw}
    \paragraph{PART I: PROBELM SET}
    \section{Reinforcement Learning I}
    The reward does not communicate the goal to the robot well. The robot can only get a positive reward at the end of the maze while all the other points have 0 reward. That is to say, the robot does not know what to do until it first reaches the exit and it keeps wandering around before that. A better reward would be giving all the failed runs a negative reward so that the robot would know that it should try to navigate to the exit as quick as possible.
    \section{Reinforcement Learning II}
\textbf{(a)} The signs do not matter in continuing tasks. For episodic tasks, however, the sign of these rewards do matter as in problem I.\\\\
\textbf{(b)} Based on the fact that:
\[R_{t} = \sum_{k=0}^{\infty} \gamma^{k} r_{t+k+1}\]
Adding a constant C to all the rewards will simply yield a new reward $\tilde{R_{t}}$ such that:
\[\tilde{R_{t}} = \sum_{k=0}^{\infty} \gamma^{k} (r_{t+k+1} + C)\]
which is also equivalent to:
\[\sum_{k=0}^{\infty} \gamma^{k} (r_{t+k+1}) + \sum_{k=0}^{\infty} C \gamma^k\]
That is to say:
\[\tilde{R_{t}} = R_{t} + \sum_{k=0}^{\infty} C \gamma^k \]
Putting this in to the Value function $V^{\pi}(s) = \mathbb{E}_{\pi} [R_{t}\mid s_t =s]$\\
We can easily obtain the fact that 
\[\tilde{V}_{\pi}(s) = \mathbb{E}_{\pi} [G_{t} + \sum_{k=0}^{\infty} \gamma^k C \mid S_t = s] = V_{\pi} (s) + \sum_{k=0}^{\infty} \gamma^k C\]
Since the discounted factor $\gamma$ is always smaller than 1, the second term in the equation above can be simplified as $\frac{C}{1-\gamma}$ based on the geometric series sum.\\\\
\textbf{(c)} Based on the facts above, that is to say, the constant K added to the values of all states is simply 
\[\frac{C}{1-\gamma}\]
\end{document}